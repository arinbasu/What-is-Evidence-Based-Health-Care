The purpose of this document is to outline the key concepts of evidence based health and medicine. We begin with a description of what is meant by evidence based health care and why we should study this topic. Then we outline the steps of evidence based practices in health and health care. 

Evidence based health care refers to a process of judicious use of best available evidence for the purpose of making health care decisions for the benefit of the individual patient, keeping in perspective the clinician's or practitioner's own expertise and the patient's or the client's requirements and values. The practices include five different steps. It starts with the step of formulating a problem statement in terms of the person, intervention or exposure under consideration, comparator conditions, and outcomes that are of interest to the patient or client. Following this initial step of formulation of the key quesiton, the evidence based practitioner (EBP) then formulates a search strategy and initiates a search of the key literature databases. The liteature databases provide the practiitioner with sources of literature. In the third step, the pracitioner appraises the available evidence to match two issues: is the body of the available evidence valid? And secondly, is the available evidence applicable to the patient or client or the situation at hand? In the fourth and final step, the evidence based pracitiioner then applies the appraised and summarised evidence to the patient or client after taking into consideration the patient or the client preferences and his or her own previous experiences. 

This approach towards addressing health care problem solving or health care practice is a major change from the traditional practice in which medical care has been 
